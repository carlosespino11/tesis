%%% fokker.tex

\chapter{Prueba $\chi^2$ de Pearson}
\label{apchitest}

La prueba $\chi^2$ de Pearsones una prueba estadística aplicada a datos categóricos para evaluar qué tan probable es que la diferencia conteos entre categorías es dada de manera aleatoria o no.

Consideremos $r$ categorias, podemos construir una tabla insertando en cada fila $i$ celdas para el conteo esperado $O_{i}$ y el conteo esperado $E_i$ de la categoría $i$. Dicha tabla recibe el nombre de \textit{tabla de contingencia}.

Sea $H_0$ la hipótesis de que las frecuencias observadas por categoría provienen de un proceso aleatorio, es decir,
\begin{equation}
 H_0: E_{i}=O_i, 
\end{equation}
el valor del estadístico de prueba es
\begin{equation}
\chi^2 = \sum_{i=1}^n \dfrac{\left( O_i-E_i \right)^2}{E_i}.
\end{equation}

Dicha prueba fue propuesta por \citet{pearson90}. Dicho estatístico alcanza su valor más bajo cuando $O_i=E_i$. Para un tamaño de muestra fijo, valores altos en $O_i-E_i$ producen $\chi^2$ grandes y por lo tanto, mayor evidencia contra $H_0$. Cuando $n\rightarrow \infty$, el estadístico $\chi^2 \sim \chi^2_{(r-1)(c-1)}$.
No se rechaza $H_0$ cuando $\chi^2 < \chi^2_{(1-\alpha)} {(r-1)(c-1)}$  donde $\alpha$ es la significancia estadística, $c$ el número de columnas  y $r$ el número de filas de la tabla de contingencia.
% etc

% example figure:

% \begin{figure}[t]
% \centerline{\epsfig{figure=fig1.eps,width=5in}}
% \caption[Probability density on the surface $\Omega(E,t)$]{
% Projecting a probability density on the surface $\Omega(E,t)$
% down to $\rho(E,t)$ in the $E-t$ plane,
% or across to $\eta(\Omega,t)$ in the $\Omega-t$ plane.
% The gradient $g$ is also shown.
% }
% \label{fig:fokker}
% \end{figure}


% We assume a one-to-one time-dependent mapping $\Omega = \Omega(E,t)$,
% which can be represented as a (fixed) surface in
% three-dimensional $(E,t,\Omega)$ space (Fig. \ref{fig:fokker}).
%  $\cdots$



 %%%
