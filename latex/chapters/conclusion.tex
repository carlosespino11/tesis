\chapter{Conclusiones}
Después de haber realizado el análisis de conglomerados y las pruebas de autocorrelación espacial, se obtuvieron los siguientes resultados:

\begin{itemize}

\item Las pruebas sobre los índices $\mathcal{I}$ y $\mathcal{C}$ mostraron alta autocorrelación espacial positiva en el índice de marginación. Cómo se observó en el mapa \ref{obj:mapmarg} hay aglomeraciones muy marcadas. Las zonas más marginadas corresponden principalmente a municipios con población indígena y de difícil acceso.

\item A partir del análisis de conglomerados esférico sobre las variables de marginación, se encontró una estructura espacial latente entre los municipios:

\begin{itemize}
\item Dentro del análisis de conglomerados, el estadístico Gap mostró que 5 es un número óptimo de grupos y se utilizó el algoritmo de $k$-medias esféricas para hacer los conglomerados. En el primer grupo cayeron municipios con localidades de pocos habitantes y con grado marginación de media a bajo; en el segundo, cayeron los municipios más marginados cuyo principal rasgo es la carencia de agua entubada; el tercer grupo tiene municipios con grado de marginación de medio a alto, lo que lo separa del grupo dos es que tiene mayor porcentaje de viviendas con agua entubada; en el cuarto, se agruparon los municipios con menor grado de marginación; y en el quinto, cayeron municipios con características similares al del primer grupo pero se diferencia en que cuenta con localidades más grandes.

\item Para comprobar la autocorrelación espacial positiva de los grupos obtenidos, se utilizaron los estadísticos $N_{ss}$ de conteo de fronteras. Los conteos entre fronteras del mismo grupo son significativamente mayores a los conteos esperados, indicando un grado de asociación espacial alto.
\end{itemize}
\end{itemize}

La importancia de este estudio está en que nos permite identificar aglomeraciones en el mapa y conocer las necesidades de éstas. Esto permite definir estrategias en materia de infraestructura para poder atender las carencias o necesidades de cada uno de los municipios. Por ejemplo, la instalación de centros de salud o de atención en una zona céntrica en la Sierra Tarahumara. 

Es importante señalar que la marginación de un municipio podría estar correlacionada con otras variables, como la dificultad de acceso o las condiciones geográficas del municipio.

\section*{Otros enfoques posibles}
\begin{itemize}
\item Podría realizarse un análisis similar para identificar focos rojos de violencia, necesidades en cuestión de salud e incluso para identificar segmentos de mercado por región.
\item Si quisiéramos hacer un estudio más puntual, podríamos realizar el mismo estudio a nivel AGEB (Área Geoestadística Básica) o por manzana, enfocándonos en una región específica.
\item También es posible realizar estudios espacio-temporales para ver la evolución de la marginación de los municipios a través del tiempo.
\end{itemize}







