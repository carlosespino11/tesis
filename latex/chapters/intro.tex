\chapter{Introducción}

Este trabajo tiene cuatro objetivos: presentar una introducción al análisis de conglomerados y comparar dos algoritmos, dando enfoque al algoritmo de $k$-medias esférico; dar una introducción del análisis estadístico espacial y presentar algunas pruebas de autocorrelación espacial; hacer una análisis exploratorio espacial del índice de marginación en México; y por último, presentar una aplicación, agrupando municipios de México utilizando variables de marginación y probando autocorrelación espacial positiva entre los grupos formados. Existen algunos algoritmos de conglomerados que utilizan la estructura espacial de los datos para formar los grupos. Sin embargo, este trabajando pretende detectar una estructura espacial latente entre los municipios de México utilizando solamente sus medidas de marginación, sin información espacial, para generar los grupos.

El análisis de conglomerados tiene como objetivo agrupar objetos, tomando como base solamente la información que encontramos en los datos que describen al objeto y a sus relaciones. Para dicho análisis se presentan los algoritmos de $k$-medias tradicional y $k$-medias esférico; este último utiliza la distancia de cosenos entre observaciones para medir la similitud entre observaciones. Así, si normalizamos las observaciones, éstas quedan sobre una esfera de radio 1 y basta con calcular el producto punto para medir el coseno del ángulo formado entre 2 observaciones. Para estimar el número óptimo de grupos, se presenta el estadístico Gap, propuesto por \citet{tibshirani01}.

Adicionalmente, se da una introducción al análisis estadístico espacial y al problema de la autocorrelación espacial. El término ``autocorrelación'' se refiere a la correlación de una variable consigo misma, en este caso, sobre el espacio. El estudio de la estadística espacial toma diferentes formas de acuerdo al tipo de datos utilizados. En consecuencia, se presentan los estadísticos $\mathcal{I}$ de Moran y $\mathcal{C}$ de Geary para datos numéricos, y los estadísticos de conteo de fronteras para datos nominales. También se muestran algunas formas para hacer pruebas de significacncia estadística sobre dichos estadísticos, ya sea utilizando el supuesto de normalidad o utilizando simulaciones de Monte Carlo.

Por último, se presenta una aplicación sobre la base de datos de CONAPO ``Índice de Marginación por Entidad Federativa y Municipio 2010''. Primero, se utilizan los índices $\mathcal{I}$ y $\mathcal{C}$ para comprobar autocorrelación espacial positiva sobre el índice de marginación. En segundo lugar, se utiliza el algoritmo de $k$-medias esféricas sobre las variables indicadoras de marginación y escogiendo el número de grupos a través del estadístico Gap. Por último, se utilizan los estadísticos de conteo de fronteras ra corroborar la auotocorrelación espacial positiva de los grupos formados, comprobando así, la homogeneidad de éstos en el espacio.

El presente trabajo se divide en 5 capítulos además de esta introducción. El segundo capítulo introduce los fundamentos del análisis de conglomerados y presenta los algoritmos de $k$-medias esféricas. El tercer capítulo da una introducción del análisis estadístico espacial. El cuarto capítulo define el término de autocorrelación espacial presentando los índices y pruebas utilizados. En quinto capítulo se exponen los resultados obtenidos al hacer el análisis de conglomerados y las pruebas de autocorrelación espacial. Por último, el sexto capítulo contiene las conclusiones y posibles aplicaciones.