%% frontmatter.tex
%%
\title{Análisis de Conglomerados Esférico para Comprobar Autocorrelación Espacial Positiva: Una Aplicación al Índice de Marginación en México}
\author{Carlos Espino García}
\degreemonth{Febrero}
\degreeyear{2015}
\degree{Licenciado en Matemáticas Aplicadas}
\field{Matemáticas Aplicadas}
\department{Matemáticas y Estadística}
\advisor{Dr. Juan José Fernández Durán } 
\place{México, D.F.}

\maketitle
% \maketitle
% \copyrightpage
\makeitamdoc




%We show how a systematic subtraction of the `special' components of a general
%deformation can be used to give an improved
%version of the `wall formula' estimate for $\mu(0)$.
%We believe this is the first study of $\omega$-dependent heating rate in
%billards, and the first consideration of the `special' nature of dilation.





% these are optional in the Jan 2000 Harvard thesis GSAS guide:
%\listoffigures
%\listoftables
%(Cut them for my personal thesis format).

% cccccccccccccccccccccccccccccccccccccccccccccccccccccccccccccccccccccccccc
% \begin{citations}

% \vspace{0.8in}

% \ssp
% \noindent
% % Large portions of Chapters~\ref{ch:dil} and \ref{ch:wall}, as well as some
% % of Sections~\ref{sec:qcc} and \ref{sec:quasi}
% have appeared in the following two papers:
% \begin{quote}
% 	``Deformations and dilations of chaotic billiards:
% 	dissipation rate, and quasi-orthogonality of the boundary
% 	wavefunctions'',
% 	A. H. Barnett, D. Cohen, and E. J. Heller,
% 	Phys. Rev. Lett. {\bf 85}, 1412 (2000), {\tt nlin.CD/0003018};
% 	\vspace{.1in} \\
% 	``Rate of energy absorption for a driven chaotic cavity'',
% 	A. H. Barnett, D. Cohen, and E. J. Heller,
% 	submitted to J. Phys. A, {\tt nlin.CD/0006041}.
% \end{quote}
% The numerical methods of
% % Chapter~\ref{ch:verg} were used to calculate data appearing in the above
% papers and in the following:
% \begin{quote}
% 	``Parametric evolution for a deformed cavity'',
% 	D. Cohen, A. H. Barnett, W. Bies, and E. J. Heller,
% 	submitted to Phys. Rev. E, {\tt nlin.CD/0008040}.
% \end{quote}
% % Chapter~\ref{ch:qpc} appears in its entirety as
% \begin{quote}
% 	``Mesoscopic scattering in the half-plane:
% 	how much conductance can you squeeze through a small hole?'',
% 	A. H. Barnett, M. Blaauboer, A. Mody, and E. J. Heller,
% 	submitted to Phys. Rev. B,
% 	{\tt cond-mat/0008279}.
% \end{quote}
% % Finally, most of Chapter~\ref{ch:atom} has been published as
% \begin{quote}
% 	``Substrate-based atom waveguide using guided two-color
% 	evanescent light fields'',
% 	A. H. Barnett, S. P. Smith, M. Olshanii, K. S. Johnson,
% 	A. W. Adams, M. Prentiss,
% 	Phys. Rev. A {\bf 61}, 023608 (2000), {\tt physics/9907014}.
% \end{quote}
% Electronic preprints (shown in {\tt typewriter font}) are available
% on the Internet at the following URL:
% \begin{quote}
% 	{\tt http://arXiv.org}
% \end{quote}
% \end{citations}

\dedication
% Dedicado a mis papás y a mis amigos
\begin{quote}
\hsp
\em
\raggedleft

Para mis papás, Ida y Carlos\\
y mi hermana, Ida.

\end{quote}


\begin{acknowledgments}

Esta tesis representa la conclusión de una de las mejores etapas de mi vida. Quiero expresar mi agradecimiento a todas las personas que han formado parte de esta gran experiencia como estudiante en el ITAM.

Antes que nada, gracias a Dios por todas las bendiciones que he recibido en la vida, por ser mi fortaleza en momentos de debilidad y por brindarme una vida llena de aprendizajes, experiencias y felicidad.

Gracias a mi familia por estar siempre a mi lado. A mis papás, Ida y Carlos, por todo el apoyo, el cariño, los valores que me han inculcado y por haberme dado la oportunidad de tener una excelente educación. A mi hermana, Ida, por ser siempre un gran ejemplo y por inspirarme a estudiar tan increíble carrera. A mi cuñado Mau, por ser como un hermano para mí.

Gracias a mi asesor, Juan José Fernández Durán, por todas las horas dedicadas asesorando esta tesis, por sus enseñanzas, su paciencia y sus consejos.

Gracias a mis sinodales Rubén Hernández, Alberto Tubilla y Fernando Esponda por sus comentarios y revisiones que ayudaron a enriquecer este trabajo.

Muchas gracias a todos mis profesores del ITAM por todas sus enseñanzas que me trajeron hasta aquí. Especialmente quiero agradecer a mis maestros de matemáticas y estadística Guillermo Grabinsky, Ramón Espinosa, Gustavo Preciado, Juan Carlos Aguilar, Victor Guerrero, Manuel Mendoza, Juan Jose Fernández Durán, Rubén Hernández, Luis Felipe González y Luis García Naranjo por reforzar mi gusto por las matemáticas. A mis profesores de computación Fernando Esponda y Silvia Guardati por hacerme disfrutar la programación. A mis profesores de estudios generales Alfredo Villafranca y Margarita Aguilera por enseñarme que no sirve de nada lo que haga en la vida si no está basado en mejorar nuestro entorno y en ayudar a resolver los problemas de nuestra sociedad.

Gracias a mis amigos de toda la vida, Oso, Rodrigo, Gordo, Jeringa, Mañon, Gorgi y Andrés por todos los momentos que hemos vividos y porque sé que pase lo que pase siempre puedo contar con ustedes.

Gracias a mis amigos del ITAM por confiar y creer en mí y haber hecho de mi etapa universitaria una experiencia que jamás olvidaré.

Gracias a Sofía, Julián, Celina, Pau y Chonki por ser parte significativa en mi vida, sé que estos años apenas son el principio de una larga amistad.

Gracias Ame, Raúl, Nico, Hans, Oscarín, Jimmy, Juanpi, Linda y Andrea, sin ustedes tantas clases y tantas horas de estudio no hubieran sido tan divertidas. Sé que su amistad me la llevo para toda la vida.

Gracias a Guillermo Garduño por ofrecerme mi primer trabajo en Sinnia. A Elmer Garduño y Rodrigo Fortes por ser mis maestros fuera del aula de clases, y a mis amigos Tania, Sergio, Andrea, Maru, Yuriko, Areli y Sonia.

\end{acknowledgments}





% \begin{abstract}


% \end{abstract}


\newpage
\addcontentsline{toc}{section}{Índice}
\tableofcontents


\newpage

\startarabicpagination

%%% end

